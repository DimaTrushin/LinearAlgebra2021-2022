\ProvidesFile{lecture04.tex}[Лекция 4]


\begin{claim}
Пусть $S_1\in\MatrixDim{m}{n}$ и $S_2\in\MatrixDim{k}{n}$ -- произвольные матрицы в улучшенном ступенчатом виде.
Если $S_1x = 0$ эквивалентно $S_2x=0$, то $S_1 = S_2$.
\end{claim}
\begin{proof}
Так как $S_1 x = 0$ и $S_2x = 0$ эквивалентны между собой, то если мы возьмем любое уравнение $l$ из системы $S_1 x = 0$ и добавим его к системе $S_2 x = 0$, получив систему $\left(\frac{S_2}{l}\right)x=0$, то новая система будет эквивалентна всем трем.
Аналогично, можно перекладывать уравнения из второй системы в первую, не меняя множества решений.

Пусть для определенности матрицы $S_1$  и $S_2$ имеют следующий вид:
\[
S_1 = 
\begin{pmatrix}
{1}&{*}&{0}&{*}&{0}&{0}&{*}&{*}&{*}\\
{}&{}&{1}&{*}&{0}&{0}&{*}&{*}&{*}\\
{}&{}&{}&{}&{1}&{0}&{*}&{*}&{*}\\
{}&{}&{}&{}&{}&{1}&{*}&{*}&{*}\\
\end{pmatrix}\quad
S_2 = 
\begin{pmatrix}
{1}&{\bullet}&{\bullet}&{0}&{\bullet}&{\bullet}&{0}&{\bullet}&{\bullet}\\
{}&{}&{}&{1}&{\bullet}&{\bullet}&{0}&{\bullet}&{\bullet}\\
{}&{}&{}&{}&{}&{}&{1}&{\bullet}&{\bullet}\\
\end{pmatrix}
\]
Они вообще говоря могут содержать разное количество ненулевых строк, пока мы ничего про это не знаем.

Давайте докажем, что в системах совпадают последние уравнения, потом следующие и так далее.
Будем двигаться снизу вверх от коротких к более длинным.
Нам надо показать три вещи: почему совпадают самые короткие уравнения, объяснить как показать совпадение для произвольного промежуточного уравнения и почему у одной из системы уравнения не закончатся раньше, чем у другой.

Пусть для определенности последнее уравнение $S_2$ не длиннее последнего уравнения $S_1$, как на картинке.
Добавим это уравнение к системе $S_1$.
Тогда возможны два случая: уравнение либо строго короче, либо имеет такую же длину.
В первом случае получим две эквивалентные системы с матрицами
\[
S_1 = 
\begin{pmatrix}
{1}&{*}&{0}&{*}&{0}&{0}&{*}&{*}&{*}\\
{}&{}&{1}&{*}&{0}&{0}&{*}&{*}&{*}\\
{}&{}&{}&{}&{1}&{0}&{*}&{*}&{*}\\
{}&{}&{}&{}&{}&{1}&{*}&{*}&{*}\\
\end{pmatrix}\quad
S_1' = 
\begin{pmatrix}
{1}&{*}&{0}&{*}&{0}&{0}&{*}&{*}&{*}\\
{}&{}&{1}&{*}&{0}&{0}&{*}&{*}&{*}\\
{}&{}&{}&{}&{1}&{0}&{*}&{*}&{*}\\
{}&{}&{}&{}&{}&{1}&{*}&{*}&{*}\\
{}&{}&{}&{}&{}&{}&{1}&{\bullet}&{\bullet}\\
\end{pmatrix}\quad
\]
Но по предыдущему утверждению это невозможно.
Значит уравнения имеют одинаковую длину, потому эквивалентны системы
\[
S_1 = 
\begin{pmatrix}
{1}&{*}&{0}&{*}&{0}&{0}&{*}&{*}&{*}\\
{}&{}&{1}&{*}&{0}&{0}&{*}&{*}&{*}\\
{}&{}&{}&{}&{1}&{0}&{*}&{*}&{*}\\
{}&{}&{}&{}&{}&{1}&{*}&{*}&{*}\\
\end{pmatrix}\quad
S_1' = 
\begin{pmatrix}
{1}&{*}&{0}&{*}&{0}&{0}&{*}&{*}&{*}\\
{}&{}&{1}&{*}&{0}&{0}&{*}&{*}&{*}\\
{}&{}&{}&{}&{1}&{0}&{*}&{*}&{*}\\
{}&{}&{}&{}&{}&{1}&{*}&{*}&{*}\\
{}&{}&{}&{}&{}&{1}&{\bullet}&{\bullet}&{\bullet}\\
\end{pmatrix}\quad
\]
В матрице $S_1'$ вычтем предпоследнее уравнение из последнего.
Новая система  $S_1''x=0$ будет эквивалентна $S_1x =0$.
Если уравнения не совпадают, то разность даст новую ступеньку и по предыдущему утверждению системы не могут быть эквивалентными.
Значит последние уравнения совпадают.

Теперь мы знаем, что матрицы $S_1$ и $S_2$ имеют вид (где треугольниками отмечены элементы одинаковых строк):
\[
S_1 = 
\begin{pmatrix}
{1}&{*}&{0}&{*}&{0}&{0}&{*}&{*}&{*}\\
{}&{}&{1}&{*}&{0}&{0}&{*}&{*}&{*}\\
{}&{}&{}&{}&{1}&{0}&{*}&{*}&{*}\\
{}&{}&{}&{}&{}&{1}&{\blacktriangle}&{\blacktriangle}&{\blacktriangle}\\
\end{pmatrix}\quad
S_2 = 
\begin{pmatrix}
{1}&{\bullet}&{\bullet}&{0}&{\bullet}&{0}&{\bullet}&{\bullet}&{\bullet}\\
{}&{}&{}&{1}&{\bullet}&{0}&{\bullet}&{\bullet}&{\bullet}\\
{}&{}&{}&{}&{}&{1}&{\blacktriangle}&{\blacktriangle}&{\blacktriangle}\\
\end{pmatrix}
\]
Теперь посмотрим на следующую пару уравнений.
Пусть для определенности уравнение в $S_1$ будет не длиннее, чем уравнение в $S_2$.
Добавим второе уравнение из $S_1$ в $S_2$ и получим эквивалентную систему.
У нас как и выше два варианта: либо длина уравнения строго меньше, либо длины одинаковые.
Рассмотрим случай первый:
\[
S_2' = 
\begin{pmatrix}
{1}&{\bullet}&{\bullet}&{0}&{\bullet}&{0}&{\bullet}&{\bullet}&{\bullet}\\
{}&{}&{}&{1}&{\bullet}&{0}&{\bullet}&{\bullet}&{\bullet}\\
{}&{}&{}&{}&{1}&{0}&{*}&{*}&{*}\\
{}&{}&{}&{}&{}&{1}&{\blacktriangle}&{\blacktriangle}&{\blacktriangle}\\
\end{pmatrix}\quad
S_2 = 
\begin{pmatrix}
{1}&{\bullet}&{\bullet}&{0}&{\bullet}&{0}&{\bullet}&{\bullet}&{\bullet}\\
{}&{}&{}&{1}&{\bullet}&{0}&{\bullet}&{\bullet}&{\bullet}\\
{}&{}&{}&{}&{}&{1}&{\blacktriangle}&{\blacktriangle}&{\blacktriangle}\\
\end{pmatrix}
\]
В этом случае по предыдущему утверждению системы не эквивалентны, чего быть не может.
Значит у нас второй случай:
\[
S_2' = 
\begin{pmatrix}
{1}&{\bullet}&{\bullet}&{0}&{\bullet}&{0}&{\bullet}&{\bullet}&{\bullet}\\
{}&{}&{}&{1}&{\bullet}&{0}&{\bullet}&{\bullet}&{\bullet}\\
{}&{}&{}&{1}&{*}&{0}&{*}&{*}&{*}\\
{}&{}&{}&{}&{}&{1}&{\blacktriangle}&{\blacktriangle}&{\blacktriangle}\\
\end{pmatrix}\quad
S_2 = 
\begin{pmatrix}
{1}&{\bullet}&{\bullet}&{0}&{\bullet}&{0}&{\bullet}&{\bullet}&{\bullet}\\
{}&{}&{}&{1}&{\bullet}&{0}&{\bullet}&{\bullet}&{\bullet}\\
{}&{}&{}&{}&{}&{1}&{\blacktriangle}&{\blacktriangle}&{\blacktriangle}\\
\end{pmatrix}
\]
Как и раньше, в $S_2'$ вычтем из нового уравнения вышестоящее.
Предположим, что уравнения были разные и получилась ненулевая строка.
Вопрос: где не может начинаться эта строка?
Ответ: там, где у обеих строк были нули.
Теперь воспользуемся тем, что все нижестоящие уравнения у нас одинаковые.
Это значит, что нули у обеих строк в одних и тех же местах (это места где начинаются нижестоящие строки).
Значит, может получится что-то вроде
\[
S_2'' = 
\begin{pmatrix}
{1}&{\bullet}&{\bullet}&{0}&{\bullet}&{0}&{\bullet}&{\bullet}&{\bullet}\\
{}&{}&{}&{1}&{\bullet}&{0}&{\bullet}&{\bullet}&{\bullet}\\
{}&{}&{}&{}&{*}&{0}&{*}&{*}&{*}\\
{}&{}&{}&{}&{}&{1}&{\blacktriangle}&{\blacktriangle}&{\blacktriangle}\\
\end{pmatrix}\quad\text{или}\quad
S_2'' = 
\begin{pmatrix}
{1}&{\bullet}&{\bullet}&{0}&{\bullet}&{0}&{\bullet}&{\bullet}&{\bullet}\\
{}&{}&{}&{1}&{\bullet}&{0}&{\bullet}&{\bullet}&{\bullet}\\
{}&{}&{}&{}&{}&{}&{*}&{*}&{*}\\
{}&{}&{}&{}&{}&{1}&{\blacktriangle}&{\blacktriangle}&{\blacktriangle}\\
\end{pmatrix}\quad\text{и т.д.}
\]
Но по предыдущему утверждению такого опять быть не может, так как новая система не эквивалентна $S_2 x = 0$.
Продолжая аналогично, мы показываем, что все уравнения у систем совпадают.

Осталось объяснить почему уравнения в одной из систем не могут закончиться раньше, чем в другой.
Но тогда у нас они обе в ступенчатом виде и одна получена из другой добавлением нескольких уравнений.
Добавление одного уменьшает множество решений, как показано в предыдущем утверждении, а добавление нескольких -- тем более.
\end{proof}

Из этого утверждения следует, что матрица улучшенного ступенчатого вида для любой матрицы $A\in\MatrixDim{m}{n}$ определена однозначно.
Так как если матрица $A$ приводится к двум разным ступенчатым видам, то их однородные системы эквивалентны, а значит они совпадают.
Потому, говоря о матрице $A$, можно говорить и о ее улучшенном ступенчатом виде без какой-либо неоднозначности.

\paragraph{Классификация}

\begin{claim}
Пусть $A,B\in\MatrixDim{m}{n}$ и пусть $E_A, E_B\subseteq \mathbb R^n$ -- множества решений систем $Ax = 0$ и $Bx = 0$, соответственно.
Тогда следующее эквивалентно:
\begin{enumerate}
\item $E_A = E_B$, т.е. системы эквивалентны.

\item $A$ приводится к $B$ элементарными преобразованиями строк.

\item Существует обратимая $C\in\Matrix{m}$ такая, что $B = CA$.

\item Матрица улучшенного ступенчатого вида для $A$ совпадает с матрицей улучшенного ступенчатого вида для $B$.
\end{enumerate}
\end{claim}
\begin{proof}
Мы все это уже доказали по сути, потому напомним, что откуда следует.
(2)$\Rightarrow$(1) Так как элементарные преобразования меняют систему на эквивалентную.
(1)$\Rightarrow$(4)  Предыдущее утверждение.
(4)$\Rightarrow$(2) Если матрицы $A$ и $B$ приводятся элементарными преобразованиями к одной и той же матрице (улучшенного ступенчатого вида), то они переводятся и друг в друга.
Эквивалентность (2)$\Leftrightarrow$(3) следует из Утверждения~\ref{claim::InvertibleDiscription} о том, что матрица обратима тогда и только тогда, когда она раскладывается в произведение элементарных.
\end{proof}

Смысл этого утверждения в следующем.
Возьмем множество всех однородных систем фиксированного размера, которое описывается матрицами $\MatrixDim{m}{n}$.
Тогда на этом множестве есть отношение эквивалентности: системы эквиваленты если они имеют одинаковое множество решений.
Это полезное свойство, потому что нам не важно какую из систем решать среди эквивалентных.
Однако, это свойство сложно проверяется.
С другой стороны, у нас есть процедура изменения системы (элементарные преобразования), которая меняет системы на заведомо эквивалентные.
Сделаем следующие замечания:
\begin{enumerate}
\item Утверждается, что эта процедура эффективная в том смысле, что если уж какие-то системы были эквивалентны, то мы обязательно от одной к другой сможем перейти элементарными преобразованиями.

\item Все то же самое верно и для второй процедуры -- умножение на обратимую матрицу слева (потому что это по сути та же самая процедура).

\item Утверждается, что в каждом классе эквивалентных систем мы можем найти одну единственную матрицу улучшенного ступенчатого вида.
То есть классов попарно неэквивалентных систем ровно столько же, сколько матриц улучшенного ступенчатого вида.

\item Последнее означает, что свойства системы с произвольной матрицей точно такие же, как у какой-то системы в улучшенном ступенчатом виде.
Потому в абстрактных задачах про системы можно всегда предполагать, что система уже имеет улучшенный ступенчатый вид.
\end{enumerate}

\subsection{Полиномиальное исчисление от матриц}

Обозначим множество всех многочленов с вещественными коэффициентами через $\mathbb R[x]$.
Формально это значит: $\mathbb R[x]=\{a_0+a_1x + \ldots + a_n x^n\mid n\in \mathbb Z_{+},\, a_i\in \mathbb R\}$.
Аналогично можно обозначать многочлены с рациональными, целыми, комплексными и т.д. коэффициентами.


\paragraph{Подстановка матриц в многочлены}

Пусть $p(x) = a_0+a_1x+\ldots a_n x^n$ -- многочлен с вещественными коэффициентами, а $A\in\Matrix{n}$.
Тогда можно определить $f(A)= a_0 E + a_1 A^1 + \ldots + a_n A^n\in\Matrix{n}$.
Если определить $A^0 = E$, то формула становится более единообразной $f(A)= a_0 A^0 + a_1 A^1 + \ldots + a_n A^n$.
Однако, психологически проще думать так: вместо $x$ подставляем $A$, а свободный член отождествляем со скалярными матрицами.
Отметим, что если два многочлена равны, то и их значения на матрице $A$ тоже равны.

\begin{claim*}
Пусть $A\in\Matrix{n}$ и $f,g\in\mathbb R[x]$ -- два произвольных многочлена, тогда:
\begin{enumerate}
\item $(f+g)(A) = f(A) + g(A)$.

\item $(fg)(A) = f(A)g(A)$.

\item $f(\lambda E) = f(\lambda)E$.

\item $f(C^{-1}AC) = C^{-1}f(A)C$ для любой обратимой $C\in \Matrix{n}$

\item Матрицы $f(A)$ и $g(A)$ коммутируют между собой.
\end{enumerate}
\end{claim*}
\begin{proof}
Все это делается прямой проверкой по определению.
Давайте объясним свойства (2) и (4).

(2) Пусть 
\[
f = \sum_{k=0}^na_k x^k\text{ и }g = \sum_{k=0}^m b_k x^k
\]
тогда 
\[
fg = \sum_{k=0}^{n + m} \left(\sum_{s + t = k }a_s b_t\right) x^k
\]
Потому надо проверить равенство:
\[
\left(\sum_{k=0}^n a_k A^k\right)\left(\sum_{k=0}^mb_k A^k\right) = \sum_{k=0}^{n+m}\left(\sum_{s+t = k} a_s b_t\right)A^k
\]
которое следует из перестановочности $A$ со своими степенями и коэффициентами.

(4) Заметим, что
\[
(C^{-1}AC)^n = C^{-1}ACC^{-1}AC\ldots C^{-1}AC = C^{-1}A^nC
\]
Осталось воспользоваться дистрибутивностью умножения, т.е. $C^{-1}(A + B)C = C^{-1}AC + C^{-1}BC$.
\end{proof}

\paragraph{Обнуляющий многочлен}

\begin{claim}
\label{claim::PolyAnnihilator}
Пусть $A\in\Matrix{n}$, тогда:
\begin{enumerate}
\item Существует многочлен $f\in\mathbb R[x]$ не равный тождественно нулю степени не больше $n^2$ такой, что $f(A) = 0$.

\item Если для какого-то многочлена $g\in\mathbb R[x]$ имеем $g(A) = 0$, а для $\lambda\in\mathbb R$ имеем $g(\lambda)\neq 0$, то $A-\lambda E$ является обратимой матрицей.
\end{enumerate}
\end{claim}
\begin{proof}
(1) Давайте искать многочлен $f$ с неопределенными коэффициентами в виде $f = a_0 + a_1 x + \ldots + a_{n^2}x^{n^2}$.
Надо чтобы было выполнено равенство $a_0 E + a_1 A + \ldots + a_{n^2}A^{n^2} = 0$.
Последнее равенство означает равенство матрицы слева нулевой матрице справа.
Это условие задается равенством всех $n^2$ ячеек матриц: $(a_0 E + a_1 A + \ldots + a_{n^2}A^{n^2})_{ij} = 0$ для всех $i,j$.
Каждое из этих условий является линейным уравнением вида $a_0 (E)_{ij} + a_1 (A)_{ij} + \ldots + a_{n^2}(A^{n^2})_{ij} = 0$.
То есть у нас есть система с $n^2$ уравнениями и $n^2 + 1 $ неизвестной.
А значит при приведении этой системы к ступенчатому виду у нас обязательно будет свободная переменная, а значит мы сможем найти ненулевое решение.

(2) Разделим многочлен $g$ на $x - \lambda$ с остатком, получим $g(x) = h(x) (x-\lambda) + g(\lambda)$.
Теперь в левую и правую часть равенства подставим $A$.
Получим 
\[
0 = g(A) = h(A)(A - \lambda E) + g(\lambda)E
\]
Перенесем $g(\lambda)E$ в другую сторону и поделим на $-g(\lambda)$, получим
\[
E = -\frac{1}{g(\lambda)}h(A)(A-\lambda E)
\]
То есть $-\frac{1}{g(\lambda)}h(A)$ является обратным к $A-\lambda E$.
\end{proof}

На самом деле можно показать, что найдется многочлен степени не больше $n$, зануляющий нашу матрицу.
Однако, мы пока не в состоянии этого сделать.

\paragraph{Спектр}

Пусть $A\in\Matrix{n}$ определим вещественный спектр матрицы $A$ следующим образом:
\[
\spec_\mathbb R A =\{\lambda \in \mathbb R\mid A - \lambda E\text{ не обратима}\}
\]
Аналогично определяются спектры в рациональном, комплексном и прочих случаях.

\begin{claim}
Пусть $A\in\Matrix{n}$ и пусть $f\in\mathbb R[x]$ такой, что $f(A) = 0$.
Тогда $|\spec_\mathbb R A|\leqslant \deg f$.
В частности спектр всегда конечен.
\end{claim}
\begin{proof}
Покажем, что любой элемент спектра является корнем $f$.
Для этого достаточно показать двойственное утверждение, если $\lambda$ не корень, то $\lambda$ не в спектре.
Но это в точности Утверждение~\ref{claim::PolyAnnihilator} пункт~(2).
\end{proof}

Так как у нас для любой матрицы найдется многочлен степени $n^2$ ее зануляющий, то спектр всегда конечен и его размер не превосходит $n^2$.
Как говорилось выше, на самом деле, можно найти многочлен степени $n$, потому спектр всегда не превосходит по мощности $n$.

\paragraph{Примеры}

\begin{enumerate}
\item Пусть $A\in\Matrix{n}$ -- диагональная матрица с числами $\lambda_1,\ldots, \lambda_n$ на диагонали, т.е.
\[
A = 
\begin{pmatrix}
{\lambda_1}&{}&{}\\
{}&{\ddots}&{}\\
{}&{}&{\lambda_n}
\end{pmatrix}
\]
Так как диагональные матрицы складываются и умножаются поэлементно
\begin{align*}
\begin{pmatrix}
{\lambda_1}&{}&{}\\
{}&{\ddots}&{}\\
{}&{}&{\lambda_n}
\end{pmatrix}
+
\begin{pmatrix}
{\mu_1}&{}&{}\\
{}&{\ddots}&{}\\
{}&{}&{\mu_n}
\end{pmatrix}
&=
\begin{pmatrix}
{\lambda_1 + \mu_1}&{}&{}\\
{}&{\ddots}&{}\\
{}&{}&{\lambda_n + \mu_n}
\end{pmatrix}\\
\begin{pmatrix}
{\lambda_1}&{}&{}\\
{}&{\ddots}&{}\\
{}&{}&{\lambda_n}
\end{pmatrix}
\begin{pmatrix}
{\mu_1}&{}&{}\\
{}&{\ddots}&{}\\
{}&{}&{\mu_n}
\end{pmatrix}
&=
\begin{pmatrix}
{\lambda_1 \mu_1}&{}&{}\\
{}&{\ddots}&{}\\
{}&{}&{\lambda_n \mu_n}
\end{pmatrix}
\end{align*}
То для любого многочлена $f\in\mathbb R[x]$ верно
\[
f
\begin{pmatrix}
{\lambda_1}&{}&{}\\
{}&{\ddots}&{}\\
{}&{}&{\lambda_n}
\end{pmatrix}
=
\begin{pmatrix}
{f(\lambda_1)}&{}&{}\\
{}&{\ddots}&{}\\
{}&{}&{f(\lambda_n)}
\end{pmatrix}
\]
То есть многочлен $f$ зануляет $A$ тогда и только тогда, когда он зануляет все $\lambda_i$.
Например, в качестве такого многочлена подойдет $f(x) = (x-\lambda_1)\ldots(x-\lambda_n)$.


Давайте покажем, что $\spec_\mathbb R A = \{\lambda_1,\ldots, \lambda_n\}$.
Так как многочлен $f$ зануляет $A$, утверждение~\ref{claim::PolyAnnihilator} пункт~(2) влечет, что спектр содержится среди его корней.
Значит, надо показать, что $A-\lambda_i E$ необратим для любого $i$.
Последнее легко видеть, так как $A-\lambda_i$ содержит $0$ на $i$-ом месте на диагонали.

\item Пусть $A=\left(\begin{smallmatrix}{0}&{-1}\\{1}&{0}\end{smallmatrix}\right)\in\Matrix{2}$.
Прямое вычисление показывает, что $A^2 = -E$, то есть многочлен $f(x) = x^2 + 1$ зануляет $A$.
Покажем, что $\spec_\mathbb R A = \varnothing$.
Действительно, по утверждению~\ref{claim::PolyAnnihilator} пункт~(2) спектр должен содержаться среди корней многочлена $f(x) = x^2 + 1$.
Однако, этот многочлен не имеет вещественных корней.
Этот пример объясняет, почему вещественных чисел иногда не достаточно и мы хотим работать с комплексными числами.
Например, в комплексном случае $\spec_\mathbb C A = \{i, - i\}$.
\end{enumerate}

\paragraph{Минимальный многочлен}

Пусть $A\in\Matrix{n}$ -- некоторая матрица.
Рассмотрим множество всех ненулевых многочленов зануляющих $A$.
Формально мы смотрим на множество
\[
M = \{f\in\mathbb R[x]\mid f(A)=0,\,f\neq 0\}
\]
Пусть $f_{min}\in M$ -- многочлен самой маленькой степени со старшим коэффициентом $1$.
Тогда он называется минимальным многочленом матрицы $A$.

\begin{claim}
\label{claim::MinPoly}
Пусть $A\in \Matrix{n}$, тогда верны следующие утверждения:
\begin{enumerate}
\item Минимальный многочлен $f_{min}$ существует.

\item Минимальный многочлен делит любой другой многочлен зануляющий $A$.

\item Минимальный многочлен единственный.

\item $\lambda\in\spec_\mathbb R A$ тогда и только тогда, когда $f_{min}(\lambda) = 0$.
\end{enumerate}
\end{claim}
\begin{proof}
(1).
По утверждению~\ref{claim::PolyAnnihilator} пункт~(1) у нас всегда найдется многочлен зануляющий $A$, а значит $M$ не пусто.
Так как степень не может убывать бесконечно, то мы обязательно найдем многочлен самой маленькой степени, который зануляет $A$.
Осталось разделить его на старший коэффициент.

(2).
Пусть $f\in M$ -- произвольный многочлен, а $f_{min}$ -- какой-то минимальный.
Тогда разделим $f$ на $f_{min}$ с остатком, получим 
\[
f(x) = h(x)f_{min}(x) + r(x)
\]
где $\deg r < \deg f_{min}$.
Подставим в это равенство матрицу $A$, получим
\[
0 = f(A) = h(A)f_{min}(A) + r(A) = r(A)
\]
Значит мы нашли многочлен $r$, который зануляет $A$ и меньше $f_{min}$ по степени.
Такое может быть только если $r(x) = 0$.

(3).
Пусть $f_{min}$ и $f'_{min}$ -- два минимальных многочлена матрицы $A$.
Тогда у них по определению одинаковая степень.
Рассмотрим $r(x) = f_{min}(x) - f'_{min}(x)$.
Многочлен $r(x)$ степени строго меньше, так как оба минимальных имеют старший коэффициент 1.
Кроме того, $r(A) = f_{min }(A) = f'_{min}(A) = 0$.
А значит $r(x) = 0$.

(4).
Мы уже знаем, что $\spec_\mathbb RA$ лежит среди корней $f_{min}$ (утверждение~\ref{claim::PolyAnnihilator} пункт~(2)).
Осталось показать обратное включение.
Предположим обратное, что есть $\lambda\in \mathbb R$ такое, что $f_{min}(\lambda) = 0$, но $\lambda\notin\spec_\mathbb RA$.
Тогда $f_{min}(x) = (x-\lambda)h(x)$.
Подставим в это равенство матрицу $A$ и получим 
\[
0 = f_{min}(A) = (A - \lambda E)h(A)
\]
Так как $\lambda\notin\spec_\mathbb R A$, то матрица $A-\lambda E$ обратима, а значит на нее можно сократить, то есть $h(A) = 0$ и степень $h$ строго меньше степени $f_{min}$, хотя сам $h$ -- ненулевой многочлен.
Последнее противоречит с нашим предположением о том, что $\lambda\notin\spec_\mathbb RA$.
\end{proof}

\paragraph{Поиск минимального многочлена}

Пусть задана матрица $A\in \Matrix{n}$.
То мы знаем, что найдется многочлен $f\in\mathbb R[x]$ такой, что $f(A) = 0$.
Кроме того, я сообщил, что $\deg f\leqslant n$.
Давайте обсудим, как найти подобный многочлен.
Будем искать его с неопределенными коэффициентами $f(x) = a_0 + a_1 x + \ldots + a_n x^n$.
Подставим в многочлен матрицу $A$ и приравняем результат к нулю.
\[
f(A) = a_0 E + a_1 A + \ldots + a_n A^n = 0
\]
Тогда то, что написано, является системой из $n^2$ уравнений, а именно
\[
\left\{_{1\leqslant i,j\leqslant n}
E_{ij}a_0 + A_{ij}a_1 + \ldots + (A^n)_{ij}a_n = 0
\right\}
\]
Здесь через $B_{ij}$ обозначены коэффициенты матрицы $B$, например, $E_{ij}$ -- это $ij$-ый коэффициент единичной матрицы, а $(A^n)_{ij}$ -- $ij$-ый коэффициент матрицы $A^n$.

Теперь нас интересует ненулевое решение этой системы, у которого как можно больше нулей справа.
Давайте поясню.
Такое решение отвечает зануляющему многочлену.
Мы хотим выбрать такой многочлен как можно меньшей степени.
То есть мы хотим по возможности занулить $a_n$, потом $a_{n-1}$, потом $a_{n-2}$ и так далее, пока находится ненулевое решение.
Предположим, что мы привели систему к ступенчатому виду и $a_k$ -- самая левая свободная переменная.
Я утверждаю, что $k$ и будет степенью минимального многочлена, а чтобы его найти надо положить $a_k = 1$, и все остальные свободные переменные равными нулю.

Действительно, если мы сделали, как описано, то все главные переменные правее $a_k$ тоже равны нулю, ибо они зависят от свободных переменных, стоящих правее, а они в нашем случае нулевые.
То есть $a_k$ будет старший ненулевой коэффициент в искомом многочлене, а значит $k$ будет его степенью.
Почему нельзя найти меньше.
Чтобы найти меньше надо занулить еще и $a_k$.
То есть все свободные переменные в этом случае будут нулевыми, а тогда и все главные будут нулевыми, а это даст нулевое решение, что противоречит нашим намерениям найти ненулевой многочлен.

\paragraph{Вычленение из какого-то зануляющего}

Предположим, что вы угадали какой-нибудь зануляющий многочлен для вашей матрицы $A\in\Matrix{n}$, а именно, нашли какой-то $f\in \mathbb R[x]$ такой, что $f(A) = 0$.
Тогда можно попытаться найти минимальный многочлен среди делителей многочлена $f$.
Эта процедура требует уметь искать эти самые делители.
Но в некоторых ситуациях эта процедура тоже бывает полезна.
Например, в случае большой блочной матрицы $A$ бывает проще найти зануляющий многочлен.

\paragraph{Замечание о спектре}

Можно показать, что любой вещественный многочлен $f\in\mathbb R[x]$ единственным образом разваливается в произведение 
\[
f(x) = (x-\lambda_1)\ldots (x-\lambda_k) q_1(x)\ldots q_r(x)
\]
где числа $\lambda_i\in\mathbb R$ могут повторяться, а $q_i(x)$ -- многочлены второй степени с отрицательным дискриминантом (то есть без вещественных корней).

Пусть теперь $f_{min}$ -- минимальный многочлен некоторой матрицы $A$.
Разложим его подобным образом.
Тогда мы видим из предыдущего утверждения, что $\spec_\mathbb RA$ помнит информацию только о первой половине сомножителей и теряет информацию о квадратичных многочленах.
Однако, если бы мы рассмотрели $f_{min}$ как многочлен с комплексными коэффициентами, то мы бы могли доразложить все $q_i(x)$ на линейные множители и $\spec_\mathbb CA$ помнит информацию о всех сомножителях $f_{min}$.
Еще надо понимать, что каждое $x-\lambda$ может несколько раз участвовать в разложении $f_{min}$, но спектр не помнит это количество, он лишь знает был ли там данный $x-\lambda$ или нет.

\paragraph{Замечание об арифметических свойствах матриц}

Если вы работаете с матрицами, то готовьтесь к тому, чтобы думать про них как про более сложную версию чисел.
А значит, вы будете писать с ними различного рода алгебраические выражения.
Например, для какой-нибудь матрицы $A\in\Matrix{n}$ можно написать $A^3 + 2 A - 3E$.
И предположим вы хотите упростить это выражение как-нибудь, не зная как именно выглядит ваша матрица $A$.
Единственное, что вам поможет в этом случае -- зануляющий многочлен.
Пусть, например, $f(x) = x^2 - 3$ зануляет $A$.
Это значит, что $A^2 = 3 E$.
Тогда выражение выше можно упростить так
\[
A^3 + 2 A - 3 E = 3A + 2 A - 3 E = 5A - 3 E
\]
Роль минимального многочлена заключается в том, что это <<самый лучший>> многочлен, который помнит как можно больше соотношений на матрицу $A$, чтобы можно было упрощать выражения.
Более того, минимальный многочлен автоматически говорит, когда можно делить на выражение от матрицы, а когда нет.
Например, на $A - E$ поделить можно, так как $1$ не является корнем $f$, с другой стороны на матрицы $A \pm\sqrt{3}E$ делить нельзя.

\paragraph{Обратимость и минимальный многочлен}

Обратимость матрицы по определению равносильна тому, что в ее спектре нет нуля, а это то же самое, что у минимального многочлена свободный член отличен от нуля.
В этом случае мы можем явно выразить обратную матрицу через исходную.
Действительно, пусть $f_{min} = a_0 + a_1 x + \ldots + a_m x^m$ для некоторой матрицы $A\in \Matrix{n}$.
Тогда
\[
a_0E + a_1 A + \ldots + a_m A^m =  0 \quad\Rightarrow\quad  A (a_1 E + \ldots + a_m A^{m-1}) =  -a_0 E \quad\Rightarrow\quad A \left(-\frac{a_1}{a_0} E - \ldots -\frac{ a_m}{a_0} A^{m-1}\right) = E
\]
То есть по определению
\[
A^{-1} = -\frac{a_1}{a_0} E - \ldots -\frac{ a_m}{a_0} A^{m-1}
\]
Обратите внимание, что данная формула работает при условии, что $a_0 \neq 0$.
Эта процедура похожа на процедуру избавления от иррациональности в знаменателе дробей или избавления от мнимой части в знаменателе в комплексных дробях.
Это не спроста, это в точности тот же самый метод.
