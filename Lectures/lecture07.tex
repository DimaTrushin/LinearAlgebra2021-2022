\ProvidesFile{lecture07.tex}[Лекция 7]


\subsection{Свойства определителя}

\paragraph{Определитель и транспонирование}

Прежде чем перейти к доказательству следующего утверждения сделаем одно полезное наблюдение.
Если мы возьмем две произвольные перестановки $\sigma,\tau\in\Sym{n}$ и матрицу $A\in\Matrix{n}$, то выражения $a_{\tau(1) \sigma(\tau (1))}\ldots a_{\tau(n) \sigma(\tau(n))}$ совпадает с выражением $a_{1\sigma(1)}\ldots a_{n\sigma(n)}$ с точностью до перестановки сомножителей.
Это делается методом пристального взгляда: замечаем что каждый сомножитель одного выражения ровно один раз встречается в другом и наоборот.

\begin{claim}
\label{claim::DetTranspose}
Пусть $A\in \Matrix{n}$, тогда $\det A = \det A^t$.
\end{claim}
\begin{proof}
Посчитаем по определению $\det A^t$, получим
\[
\det A^t = \sum_{\sigma\in\Sym{n}}\sgn(\sigma)a_{\sigma(1)1} \ldots a_{\sigma(n)n} = 
\sum_{\sigma\in\Sym{n}}\sgn(\sigma)a_{\sigma(1)\sigma^{-1}(\sigma(1))} \ldots a_{\sigma(n)\sigma^{-1}(\sigma(n))}
\]
Теперь применим наше замечание перед доказательством:
\[
a_{\sigma(1)\sigma^{-1}(\sigma(1))} \ldots a_{\sigma(n)\sigma^{-1}(\sigma(n))}
=
a_{1 \sigma^{-1}(1)}\ldots a_{n\sigma^{-1}(n)}
\]
Значит
\[
\det A^t = \sum_{\sigma\in\Sym{n}}\sgn(\sigma) a_{1 \sigma^{-1}(1)}\ldots a_{n\sigma^{-1}(n)}
\]
Вспомним, что $\sgn(\sigma) = \sgn(\sigma^{-1})$.
Следовательно:
\[
\det A^t = \sum_{\sigma\in\Sym{n}}\sgn(\sigma^{-1}) a_{1 \sigma^{-1}(1)}\ldots a_{n\sigma^{-1}(n)}
\]
Теперь, если $\sigma$ пробегает все перестановки, то $\sigma^{-1}$ тоже пробегает все перестановки, так как отображение $\Sym{n}\to \Sym{n}$ по правилу $\sigma\mapsto \sigma^{-1}$ является биекцией.%
\footnote{Оно биекция, так как имеет обратное -- оно само.}
То есть мы можем сделать замену $\tau = \sigma^{-1}$ и приходим к выражению
\[
\det A^t = \sum_{\tau\in\Sym{n}}\sgn(\tau)a_{1\tau(1)}\ldots a_{n\tau(n)}
\]
Последнее в точности совпадает с определением $\det A$.
\end{proof}

Отметим, что если мы доказали какое-то свойство определителя для столбцов, то это утверждение автоматически гарантирует, что такое же свойство выполнено и для строк.
И наоборот, если что-то сделано для строк, то это автоматом следует для столбцов.

\subsection{Полилинейность и кососимметричность определителя}

Сейчас мы докажем, что определитель обладает всеми свойствами (II) и (II').
В силу утверждения~\ref{claim::DetTranspose} нам достаточно показать только (II).

\begin{claim}
\label{claim::DetPolyAnti}
Отображение $\det\colon \Matrix{n}\to \mathbb R$ рассматриваемое как отображение столбцов матрицы является полилинейным и кососимметричным, т.е. удовлетворяет следующим свойствам:
\begin{enumerate}
\item $\det(A_1,\ldots, A_i + A_i', \ldots, A_n) = \det(A_1,\ldots, A_i, \ldots, A_n) + \det(A_1,\ldots,A_i', \ldots, A_n)$ для любого $i$.

\item $\det(A_1,\ldots, \lambda A_i, \ldots, A_n) = \lambda \det(A_1,\ldots, A_i, \ldots, A_n)$ для любого $i$ и любого $\lambda\in\mathbb R$.

\item $\det(A_1,\ldots, A_i, \ldots, A_j, \ldots, A_n) = -\det(A_1,\ldots, A_j, \ldots, A_i, \ldots, A_n)$ для любых различных $i$ и $j$.

\item $\det E = 1$.
\end{enumerate}
\end{claim}
\begin{proof}
Мы знаем, что 
\[
\det A = \det A^t = \sum_{\sigma\in\Sym{n}}\sgn(\sigma)a_{\sigma(1)1} \ldots a_{\sigma(i)i}\ldots a_{\sigma(n)n}
\]
Проверим свойство (1):
\begin{gather*}
\det(A_1,\ldots, A_i + A_i', \ldots, A_n) = \sum_{\sigma\in\Sym{n}}\sgn(\sigma)a_{\sigma(1)1} \ldots \left(a_{\sigma(i)i} + a_{\sigma(i)i}'\right)\ldots a_{\sigma(n)n}=\\
= \sum_{\sigma\in\Sym{n}}\sgn(\sigma)a_{\sigma(1)1} \ldots a_{\sigma(i)i}\ldots a_{\sigma(n)n} +\sum_{\sigma\in\Sym{n}}\sgn(\sigma)a_{\sigma(1)1} \ldots a_{\sigma(i)i}'\ldots a_{\sigma(n)n} =\\
= \det(A_1,\ldots, A_i, \ldots, A_n) + \det(A_1,\ldots,A_i', \ldots, A_n)
\end{gather*}
Теперь свойство (2):
\[
\det(A_1,\ldots, \lambda A_i, \ldots, A_n) = \sum_{\sigma\in\Sym{n}}\sgn(\sigma)a_{\sigma(1)1} \ldots \left(\lambda a_{\sigma(i)i}\right)\ldots a_{\sigma(n)n}= \lambda \det(A_1,\ldots,  A_i, \ldots, A_n) 
\]
Для проверки свойства (3) введем следующее обозначение.
Пусть $\tau\in\Sym{n}$ обозначает транспозицию $(i,j)$.
Тогда посчитаем определитель с переставленными местами столбцами $i$ и $j$:
\begin{gather*}
\det (A_1,\ldots, A_j, \ldots, A_i, \ldots, A_n) = \sum_{\sigma\in\Sym{n}}\sgn(\sigma) a_{\sigma(1)1}\ldots a_{\sigma(i)j}\ldots a_{\sigma(j)i}\ldots a_{\sigma(n)n} = \\
=  \sum_{\sigma\in\Sym{n}}\sgn(\sigma) a_{\sigma(1)\tau(1)}\ldots a_{\sigma(i)\tau(i)}\ldots a_{\sigma(j)\tau(j)}\ldots a_{\sigma(n)\tau(n)} = \\
= \sum_{\sigma\in\Sym{n}} \sgn(\sigma) a_{\sigma(\tau^{-1}(1))1} \ldots a_{\sigma(\tau^{-1}(n))n} = -\sum_{\sigma\in\Sym{n}} \sgn(\sigma\tau^{-1}) a_{\sigma(\tau^{-1}(1))1} \ldots a_{\sigma(\tau^{-1}(n))n}
\end{gather*}
Здесь при переходе от второй строчки к третьей мы воспользовались замечанием перед утверждением~\ref{claim::DetTranspose}.
Так как отображение $\Sym{n}\to \Sym{n}$ по правилу $\sigma\mapsto \sigma \tau^{-1}$ является биекцией, то если $\sigma$ пробегает все перестановки, то и $\sigma\tau^{-1}$ пробегает все перестановки.
А значит, делая замену $\rho = \sigma \tau^{-1}$, получаем
\[
-\sum_{\sigma\in\Sym{n}} \sgn(\sigma\tau^{-1}) a_{\sigma(\tau^{-1}(1))1} \ldots a_{\sigma(\tau^{-1}(n))n} = -\det (A_1,\ldots, A_i, \ldots, A_j,\ldots A_n)
\]

(4) Это непосредственно следует из определения, либо, если хотите, можно сослаться на утверждение~\ref{claim::DetUpperTr}.
\end{proof}

\begin{claim*}
Если $A\in\Matrix{n}$ имеет нулевой столбец или нулевую строку, то $\det A = 0$.
\end{claim*}
\begin{proof}
Пусть $A$ имеет нулевой столбец.
Мы знаем, что $\det$ -- полилинейная функция.
Значит, если мы умножим нулевой столбец на $-1$, определитель должен поменять знак.
С другой стороны, если мы умножим нулевой столбец на любое число, он не поменяется и определитель не должен поменяться.
Значит по безысходности определитель должен быть $0$.
\end{proof}


\paragraph{Определитель от элементарных матриц}

\begin{claim}
Верны следующие утверждения:
\begin{enumerate}
\item $\det (S_{ij}(\lambda)) = 1$, где $S_{ij}(\lambda)\in\Matrix{n}$ -- матрица элементарного преобразования первого типа.

\item $\det (U_{ij}) = -1$, где $U_{ij}\in\Matrix{n}$ -- матрица элементарного преобразования второго типа.

\item $\det(D_i(\lambda)) = \lambda$, где $D_{i}(\lambda)\in\Matrix{n}$ -- матрица элементарного преобразования третьего типа.
\end{enumerate}
\end{claim}
\begin{proof}
(1) Является следствием для случая верхне- и нижнетреугольных матриц.

(2) Так как $U_{ij}$ получается из единичной матрицы перестановкой $i$-го и $j$-го столбцов, то результат следует из кососимметричности определителя.

(3) Следует из полилинейности определителя -- свойство (II)~(2).
\end{proof}




\subsection{Полилинейные кососимметрические отображения}

Все утверждения в этом разделе доказываются для строк.
Соответствующие утверждения для столбцов доказываются аналогично.
Их формулировки и доказательства я оставляю в качестве упражнения.

\begin{claim}
\label{claim::PolyAntiAndElementary}
Пусть $\phi\colon \Matrix{n}\to \mathbb R$ -- полилинейное кососимметрическое отображение по строкам матриц, т.е. удовлетворяет следующим свойствам:%
\footnote{Здесь через $A_i$ обозначаются строки матрицы $A$ идущие сверху вниз.}
\begin{enumerate}
\item $\phi(A_1,\ldots, A_i + A_i', \ldots, A_n) = \phi(A_1,\ldots, A_i, \ldots, A_n) + \phi(A_1,\ldots,A_i', \ldots, A_n)$ для любого $i$.

\item $\phi(A_1,\ldots, \lambda A_i, \ldots, A_n) = \lambda \phi(A_1,\ldots, A_i, \ldots, A_n)$ для любого $i$ и любого $\lambda\in\mathbb R$.

\item $\phi(A_1,\ldots, A_i, \ldots, A_j, \ldots, A_n) = -\phi(A_1,\ldots, A_j, \ldots, A_i, \ldots, A_n)$ для любых различных $i$ и $j$.
\end{enumerate}
Тогда $\phi(UA) = \det(U)\phi(A)$ для любой матрицы $A\in\Matrix{n}$ и любой элементарной матрицы $U\in\Matrix{n}$.
\end{claim}
\begin{proof}
Случай $U = S_{ij}(\lambda)$.
\begin{gather*}
\phi(S_{ij}(\lambda)A) = \phi(A_1,\ldots, A_i + \lambda A_j, \ldots, A_j, \ldots, A_n) = \\
= \phi(A_1,\ldots, A_i, \ldots, A_j, \ldots, A_n) + \lambda\phi(A_1,\ldots, A_j, \ldots, A_j, \ldots, A_n) =  \phi(A) = \det(S_{ij}(\lambda))\phi(A)
\end{gather*}

Случай $U= U_{ij}$.
\begin{gather*}
\phi(U_{ij}A) = \phi(A_1,\ldots, A_j, \ldots, A_i, \ldots, A_n) = -\phi(A_1,\ldots, A_i, \ldots, A_j, \ldots, A_n) = -\phi(A) = \det(U_{ij})\phi(A)
\end{gather*}

Случай $U=D_i(\lambda)$.
\begin{gather*}
\phi(D_i(\lambda)A) = \phi(A_1,\ldots,\lambda A_i,\ldots, A_n) = \lambda \phi(A_1,\ldots, A_i,\ldots, A_n) = \lambda \phi(A) = \det(D_i(\lambda))\phi(A)
\end{gather*}
\end{proof}

\paragraph{Определитель и элементарные матрицы}

Заметим, что по утверждению~\ref{claim::DetPolyAnti}, определитель тоже является полилинейной и кососимметрической функцией.
Потому доказанное утверждение в частности означает, что $\det(UA) = \det(U)\det(A)$ для любой матрицы $A\in\Matrix{n}$ и любой элементарной матрицы $U\in\Matrix{n}$.

\paragraph{Подсчет определителя}

Предыдущее замечание позволяет дать эффективный способ вычисления определителя методом Гаусса.
Мы берем матрицу $A$ и приводим ее к ступенчатому виду, попутно запоминая как изменился определитель по сравнению с определителем изначальной матрицы.
Если же мы будем использовать только элементарные преобразования первого типа, то определитель вовсе меняться не будет.
Ступенчатый вид матрицы всегда верхнетреугольный.
Там определитель считается как произведение диагональных элементов.


\paragraph{Следствия утверждения~\ref{claim::PolyAntiAndElementary}}

\begin{claim}
[Единственность для полилинейных кососимметричных]
\label{claim::PolyAntiUnique}
Пусть $\phi\colon \Matrix{n}\to \mathbb R$ -- полилинейное кососимметрическое отображение по строкам матриц.
Тогда $\phi(X) = \det(X)\phi(E)$.
В частности, если $\phi(E) = 1$, то $\phi = \det$.
\end{claim}
\begin{proof}
Пусть $X\in\Matrix{n}$ -- произвольная матрица, тогда ее можно элементарными преобразованиями строк привести к улучшенному ступенчатому виду.
Последнее означает, что $X = U_1 \ldots U_k S$, где $S$ -- матрица улучшенного ступенчатого вида, а $U_i$ -- матрицы элементарных преобразований.
Применим к этому равенству отдельно $\phi$ и отдельно $\det$, получим
\begin{align*}
\phi(X) &= \det(U_1)\ldots \det(U_k)\phi(S)\\
\det(X) &= \det(U_1)\ldots \det(U_k)\det(S)
\end{align*}
Теперь для матрицы $S$ у нас есть два варианта: либо $S$ единичная, либо содержит нулевую строку.

Пусть $S = E$, тогда
\begin{align*}
\phi(X) &= \det(U_1)\ldots \det(U_k)\phi(E)\\
\det(X) &= \det(U_1)\ldots \det(U_k)
\end{align*}
Откуда и получаем требуемое $\phi(X) = \det(X)\phi(E)$.

Пусть теперь $S$ имеет нулевую строку.
Тогда из полилинейности определителя и $\phi$ следует, что $\phi(S) = 0 =\det(S)$.%
\footnote{При умножении на $-1$ нулевой строки с одной стороны функция должна поменять знак, а с другой не измениться.}
Что тоже влечет равенство $\phi(X) = \det(X)\phi(E)$.
\end{proof}


\begin{claim}
[Мультипликативность определителя]
\label{claim::DetMulti}
Пусть $A,B\in\Matrix{n}$ -- произвольные матрицы.
Тогда $\det(AB) = \det(A)\det(B)$.
\end{claim}
\begin{proof}
Фиксируем матрицу $B\in\Matrix{n}$ и рассмотрим отображение $\gamma\colon \Matrix{n}\to \mathbb R$ по правилу $A\mapsto \det(AB)$.
Если $A_1,\ldots, A_n$ -- строки матрицы $A$, то $A_1B, \ldots, A_nB$ -- строки матрицы $AB$.
Из этого легко видеть, что $\gamma$ -- полилинейна и кососимметрическая функция по строкам матрицы $A$.
Значит по утверждению~\ref{claim::PolyAntiUnique} $\gamma(A) = \det(A)\gamma(E)$.
Но последнее равносильно $\det(AB) = \det(A)\det (B)$.
\end{proof}

\begin{claim}
[Определитель с углом нулей]
Пусть $A,\in\Matrix{n}$ и $B\in\Matrix{m}$.
Тогда
\[
\det
\begin{pmatrix}
{A}&{*}\\
{0}&{B}
\end{pmatrix}
=
\det
\begin{pmatrix}
{A}&{0}\\
{*}&{B}
\end{pmatrix}
=
\det(A) \det(B)
\]
\end{claim}
\begin{proof}
Рассмотрим функцию $\phi\colon \Matrix{n}\to \mathbb R$ по правилу
\[
\phi(X) = 
\det
\begin{pmatrix}
{X}&{*}\\
{0}&{B}
\end{pmatrix}
\]
Заметим, что эта функция является полилинейной и кососимметричной по столбцам матрицы $X$.
В этом случае по утверждению~\ref{claim::PolyAntiUnique} о единственности для полилинейных кососимметрических отображений она имеет вид $\phi(X) = \det(X)\phi(E)$, то есть
\[
\det
\begin{pmatrix}
{A}&{*}\\
{0}&{B}
\end{pmatrix}
=
\det A
\det
\begin{pmatrix}
{E}&{*}\\
{0}&{B}
\end{pmatrix}
\]
Теперь рассмотрим функцию $\psi\colon \Matrix{m}\to \mathbb R$ по правилу
\[
\psi(X) =
\det
\begin{pmatrix}
{E}&{*}\\
{0}&{X}
\end{pmatrix}
\]
Заметим, что эта функция является полилинейной и кососимметричной по строкам матрицы $X$.
В этом случае по утверждению~\ref{claim::PolyAntiUnique} о единственности для полилинейных кососимметрических отображений она имеет вид $\psi(X) = \det(X)\psi(E)$, то есть
\[
\det
\begin{pmatrix}
{E}&{*}\\
{0}&{B}
\end{pmatrix}
=
\det B
\det
\begin{pmatrix}
{E}&{*}\\
{0}&{E}
\end{pmatrix}
\]
Последний определитель равен $1$, так как по утверждению~\ref{claim::DetUpperTr} определитель верхнетреугольной матрицы равен произведению ее диагональных элементов.
Теперь собираем вместе доказанные факты и получаем требуемый результат.
\end{proof}

Заметим, что таким образом мы можем считать определитель для блочно верхнетреугольных матриц и для блочно нижнетреугольных матриц с любым количеством блоков.
Формулы тогда будут выглядеть так
\[
\det
\begin{pmatrix}
{A_1}&{*}&{\ldots}&{*}\\
{}&{A_2}&{\ldots}&{*}\\
{}&{}&{\ddots}&{\vdots}\\
{}&{}&{}&{A_k}\\
\end{pmatrix}
=
\det
\begin{pmatrix}
{A_1}&{}&{}&{}\\
{*}&{A_2}&{}&{}\\
{\vdots}&{\vdots}&{\ddots}&{}\\
{*}&{*}&{\ldots}&{A_k}\\
\end{pmatrix}
=\det A_1 \ldots \det A_k
\]
где $A_i\in\Matrix{n_i}$ -- обязательно квадратные матрицы.
Это правило является обобщением утверждения о вычислении определителя для треугольных матриц.
